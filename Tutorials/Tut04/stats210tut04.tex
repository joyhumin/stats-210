\documentclass[12pt, oneside, a4paper]{article}
\usepackage[a4paper, left = 2cm, right = 2cm, top=3cm, bottom=2.5cm]{geometry}
\usepackage{fancyhdr}
\usepackage{graphicx}
\usepackage{mathtools}
\usepackage{amsfonts}
\usepackage{amsthm}
\usepackage{amsmath}
\usepackage{enumitem}
\usepackage{pgfplots}
\pgfplotsset{compat=newest}
\usepackage{bm}
\usepackage{gensymb}
\usepackage{pgfplotstable}
\usepackage{float}
\usepackage{tikz}
\makeatletter
\renewcommand*\env@matrix[1][*\c@MaxMatrixCols c]{%
	\hskip -\arraycolsep
	\let\@ifnextchar\new@ifnextchar
	\array{#1}}
\makeatother
\usepackage{amssymb}
\newcommand{\divides}{\mid}

%\renewcommand{\familydefault}{\sfdefault}
\renewcommand{\rmdefault}{ptm}


\pagestyle{fancy}
\fancyhf{}
\renewcommand{\headrulewidth}{0pt}
\fancyhead[C]{\thepage\\}
\fancyhead[R]{ID: 107664618 \\ Joy \underline{HU} \\ UPI: mhu138}
\begin{document}
	\begin{enumerate}
		\item \begin{enumerate}[label = (\alph*)]
			\item $X$ is the number of boys out of Garfield's seven children, it's has binomial distribution. Therefore, the distribution of $X$ is \\
			$X \sim \text{Binimoal}(7,p)$
			\item $X$ is the number of boys out of Garfield's seven children and he has $5$ sons, therefore $x = 5$
			\item The likelihood function: \\
			\begin{align*}
				L(p; 5) & = \mathbb{P}(X =5) \text{ when } X \sim \text{Binimoal}(7,p),  \\
				& = \binom{7}{5}p^5(1-p)^2 \\
				& = 21p^5(1-p)^2 \quad \text{ for }0 < p < 1
			\end{align*}
			\item $L(\frac{1}{2}; 5) = 21 \times (\frac{1}{2})^5(1-\frac{1}{2})^2 =0.164 $
			\item $L(\frac{5}{7}; 5) = 21 \times (\frac{5}{7})^5(1-\frac{5}{7})^2 = 0.319 $
				\item $L(0.8; 5) = 21 \times 0.8^5(1-0.8)^2 =  0.275$
				\item The graph is attached below
				\begin{figure}[H]
					\centering
					\includegraphics[width=0.7\linewidth]{"PNG image 2020-08-30 00_03_38"}
					\label{fig:png-image-2020-08-30-000338}
				\end{figure}
				
				
			\item \begin{align*}
				\dfrac{dL}{dp} & = 21 \times \left(5\times p^4 \times (1-p) ^2 + p^5 \times 2\times (1-p) \times (-1)\right) \quad \text{Product rule}\\
				& = 21 \times p^4 \times(1-p) \times (5\times (1-p) - 2p) \\
				& = 21p^4(1-p)(5-7p)
			\end{align*}
			\item The maximising value of $p$ occurs when $\left. \dfrac{dL}{dp}\right |_{p = \hat p} = 0 $, this gives: 
			\begin{align*}
				\left .\dfrac{dL}{dp}\right |_{p = \hat p} & = 21\hat{p}^4(1-\hat{p})(5-7\hat{p}) = 0 \\
				& \implies 5 - 7\hat{p} = 0 \\
				& \implies \hat{p} = \cfrac{5}{7}
			\end{align*}
		\item We have decided that a sensible parameter estimate for $p$ is the maximum likelihood estimate ($\hat{p}=\frac{5}{7}$): the value of $p$ at which the observation $X=5$ is more likely than at any other value of $p$.
			\end{enumerate}
		\item 
		\begin{enumerate}[label = (\alph*)]
			\item The likelihood function:\\
			\begin{align*}
			 L(p; 3) & = \mathbb{P}(X = 3 ) \text{ when }X \sim \text{Geometric}(p)\\
				L(p; 3) & = (1-p)^3p \text{ for } 0 < p < 1
			\end{align*}
			\item \begin{align*}
				\cfrac{dL}{dp} & = (1-p)^3 + p\times 3\times (1-p)^2 \times (-1) \quad (\text{by product rule})\\
				& = (1-p)^3 -3p(1-p)^2 \\
				& = (1-p)^2 (1-p-3p)\\
				& = (1-p)^2(1-4p)
			\end{align*}
			\item The miximising value of $p$ occurs when $\left.\cfrac{dL}{dp} \right |_{p = \hat{p}}= 0$, this gives
			\begin{align*}
				\left.\cfrac{dL}{dp} \right |_{p = \hat{p}}& = (1-\hat{p})^2(1-4\hat{p})  = 0\\
				 \implies 1-4\hat{p} & = 0\\
				 \hat{p} &  = \cfrac{1}{4}
			\end{align*}
			\item In common-sense, Sammie first succeeded on his fourth jump, we would think that the probability of success would be $\frac{1}{4}$. And our $MLE$ indeed same as our common-sense. 
		\end{enumerate}
	\end{enumerate}
\end{document}